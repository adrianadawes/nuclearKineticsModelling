% arara: pdflatex
\documentclass{article}
\usepackage{amsmath}
\usepackage{hyperref}

\author{Matt McDermott}
\title{Documentation for \texttt{C++} model}
\begin{document}
\maketitle
\section{Parameter Values:}
\begin{description}
  \item[\texttt{Tau}$=\tau$:] $\tau$ measures the timestep for the model and is currently
    $\frac{1}{4000}$ minutes. 
  \item[\texttt{Contact Length}:] This measures the length of time that an MT
    attached to the Cortex remains attached. It is currently
    \texttt{400*Tau}$=\frac{1}{10}$ minutes. 
  \item[\texttt{Vg}$=V_g$:] This measures the growth velocity of an MT and is
    currently $40 \mu m/\text{min}$. 
  \item[\texttt{Vs}$=V_s$:] This measures the shrinking velocity of an MT and is
    currently $120 \mu m/\text{min}$. 
  \item[\texttt{Vs\_c}$={V_s}_c$:] This measures the shrinking velocity of an MT
    post contact and is currently $120 \mu m/\text{min}$. 
  \item[\texttt{kc}$=k_c$:] This measures the catastrophe frequency of an MT and
    is currently $.1\cdot 60 1/\text{min}$. 
  \item[\texttt{kr}$=k_r$:] This measures the catastrophe frequency of an MT and
    is currently $.4\cdot 60 1/\text{min}$. 
  \item[\texttt{R1\_max}$=R_1$:] This measures the maximum length of the
    ellipsoidal cell body along the $x$-axis and is currently $50\mu m$.  
  \item[\texttt{R2\_max}$=R_2$:] This measures the maximum length of the
    ellipsoidal cell body along the $y$-axis and is currently $30\mu m$.  
  \item[\texttt{Prad}$=P_{\text{rad}}$:] This measures the radius of the
    pronucleus and is currently $5\mu m$.  
  \item[\texttt{Eta}$=\eta$:] This measures the translational drag coefficient
    of the pronucleus and is currently $1\frac{\text{pN}}{\mu m \text{min}}$. 
  \item[\texttt{Mu}$=\mu$:] This measures the translational drag coefficient
    of the pronucleus and is currently $5\eta = 5\frac{\text{pN}}{\mu m \text{min}}$. 
\end{description}
\section{Usage Instructions:}
  The three basic steps to to use this code are:\\
  \textbf{To Compile:} Run \texttt{make} on the command line.\\
  \textbf{To Run:} After compilation, run
  \begin{verbatim}
  ./mtKineticModel <Number of Runs> <Output File> [temp/all]
  \end{verbatim}
  on the command line. Here, \texttt{<Number of Runs>} denotes the number of
  runs of the given system the model should produce and \texttt{<Output File>}
  specifies the file to which the results should be written. Note that this file
  will be placed in the directory \texttt{nuclearKineticsModelling/data}. The
  final parameter, \texttt{[temp/all]}, is an optional parameter. If omitted,
  the system will only store the final configuration of the system in the result
  file. If given as \texttt{temp}, it will store 30 second time slices in the
  result system for each model run, in sequence (i.e. it will list various
  parameters for 30 second slices for the first run, then those for the second,
  and so on and so forth). If given as \texttt{all}, it will write data for
  every timestep in the result file, in sequence, for each model run. \\
  \textbf{To View: } There are a variety of matlab scripts in the
  \texttt{plottingCode} folder to visualize the results. Several key scripts are
  outlined below. All are run by setting, either in the matlab console or via
  the matlab \texttt{-r} command line flag, the variables \texttt{dataDir} and
  \texttt{dataFile} to the directory containing the data file and the name of
  the data file (without the extension), respectively, then running the script
  (to run the script, either type \texttt{run [SCRIPT NAME].m} in the matlab
  console, or at the end of the variable declarations in the \texttt{-r} command
  line flag).
  \begin{description}
    \item[\texttt{positionHeatRotationRose.m}] This script produces a heat map
      of the pronucleus' final position and a rose plot (angular histogram) of
      the pronucleus' final orientation. In addition to specifying the data
      directory and the data file via \texttt{matlab} variables \texttt{dataDir}
      and \texttt{dataFile} in this script, you can also specify the number of
      runs for which the simulation was computed. Note that this script
      \emph{only} works \& is meaningful for many simulations of the model that
      only output the default amount of data each iteration (i.e. run with
      \texttt{./mtKineticModel [NUMBER OF RUNS] [OUTPUT FILENAME]}, not 
      \texttt{./mtKineticModel [NUMBER OF RUNS] [OUTPUT FILENAME] temp} or 
      \texttt{./mtKineticModel [NUMBER OF RUNS] [OUTPUT FILENAME] all}.
    \item[\texttt{movieMaker.m}] This script produces a visualization of an
      individual run, by animating the motion of the pronucleus and MTs. It also
      writes these images to the \texttt{plottingCode/figs/} folder, which can
      then be translated into a movie using other utilities. Note that it also
      erases this \texttt{figs/} folder prior to running, so don't expect those
      images to stay around if you are running this script multiple times. In
      addition to \texttt{dataDir} and \texttt{dataFile}, you can also set the
      variables \texttt{visible} and \texttt{step} to control whether or not the
      frames are shown on the screen while it is being rendered and what the
      step size is between frames, respectively. Note that this script only
      works with single simulation runs that output all their data (i.e. run
      with \texttt{./mtKineticModel [NUMBER OF RUNS] [OUTPUT FILENAME] all}).
    \item[\texttt{centrosomePos.m}] This script produces several visualization
      of the motion of the centrosomes across time over many runs. In order to
      run this script, you additionally need to set the \texttt{numRuns}
      variable, cataloging the number of times you ran the simulation. Note that
      this script only works with runs that write data every $30$ seconds, which
      is what is produced with the \texttt{temp} data suffix. So, the data must
      be produced via
      \texttt{./mtKineticModel [NUMBER OF RUNS] [OUTPUT FILENAME] temp}.
    \item[\texttt{plotter.m}] This script produces plots examining effects
      across a large experiment produced via the \texttt{experiment.sh} script.
      For this script, rather than specifying the \texttt{dataDir} and
      \texttt{dataFile} variables, you must specify the \texttt{dataBaseDir},
      which is a directory containing all of the output from the
      \texttt{experiment.sh} script, named according to the same convention used
      by \texttt{experiment.sh}, and \texttt{figsBaseDir}, which specifies the
      base directory to which the resulting plots should be written. This
      script does not show you the plots in real time, but rather writes them to
      the \texttt{figsBaseDir} directory (which should exist prior to starting
      the script).
  \end{description}
  Again, all of these scripts are run with the following basic premise: Open
  \texttt{MATLAB}, set values \texttt{dataDir} and \texttt{dataFile}
  \textit{(and/or \texttt{dataFile2})} in the \texttt{MATLAB} console, then run
  the script. \texttt{dataDir} stands for the directory (as a sub-directory of
  \texttt{nuclearKineticsModelling/data/}) and \texttt{dataFile} stands for the
  dataFile in question, with no extension.
  \subsection{Basic Usage:}
  For basic usage, the only additional step beyond simple compilation and
  running is parameter modification. Here is a short description of some of the
  parameters we change most frequently \textit{(All found in the
  \texttt{parameters.cpp} file)}.
  \begin{description}
    \item[\texttt{springsOn} ] This boolean value controls whether the centrosomes
      are connected to the pronucleus rigidly (and thus not allowed to translate
      relative to the pronucleus) or via springs (of spring constant given later
      in the file).
    \item[\texttt{translation} ] This boolean value controls whether the
      pronucleus is allowed to translate. If it is \texttt{true}, the system
      proceeds in full generality. If it is \texttt{false}, the pronucleus
      center is fixed in the cortex (though it is allowed to rotate). If the
      springs are on, the centrosomes are also allowed to translate (and thus
      provide force via the springs to the pronucleus). 
    \item[\texttt{kM/kD} ] These parameters control the relative spring
      constants for the springs connecting the mother (\texttt{M}) centrosome and
      daughter (\texttt{D}) centrosome to the pronucleus, respectively. If
      springs are off, then these parameters do nothing. We do not yet have good
      values for these constants. 
    \item[\texttt{startX} ] This parameter controls the starting $x$ coordinate
      of the pronucleus, with $0$ being the center of the cell. We typically
      test starting at $0$ and starting at $7.5 \mu\text{m}$. 
    \item[\texttt{startPsi} ] This parameter controls the starting angular
      position of the pronucleus, in radians, with $0$ being \emph{horizontal}.
      For on-angle runs, we use $\frac{\pi}{2}$. To test the effects of angular
      variations, we typically use $\frac{\pi}{2} + k\frac{\pi}{8}$, with $1 \le
      k \le 4$, though it might be sensible to test larger $k$ and finer
      gradations. 
    \item[\texttt{numRegions/regionAngles/regionProbabilities} ] These are the
      band parameters, and control the number of uniform probability regions,
      the angular endpoints of those regions, and the associated probabilities
      with those regions. 
  \end{description}
  \subsection{Advanced Usage:}
  There are a variety of other parameters one may wish to modify, all of which
  found in the \texttt{parameters.cpp} file. For these modifications, the same
  procedure is used: modify the parameter, re-compile, and run. 
  \subsection{Major Experiments:} 
  For running a long form experiment, with many runs of many different parameter
  variations required to collect all the requisite data, manually changing the
  parameter file each time is tedious and slow. To combat this, there is also a
  script \texttt{experiment.sh} inside of \texttt{modelCode} which automates
  this process for you. Currently, experiment.sh only supports running
  experiments on the \texttt{translation}, \texttt{envWidthM},
  \texttt{envWidthD}, and \texttt{width} parameters, which correspond to the
  translation status, mother allowed-growth envelope width, daughter
  allowed-growth envelope width, and the push band width, respectively. To use
  this script, decide what values of these parameters you would like tested. The
  script will then run \emph{all possible} combinations of these values, and
  write the data to file names based on specified parameter-specific portions.
  Be warned, should the user-specified parameter naming scheme be ambiguous,
  then the script will potentially overwrite other files it has produced on a
  later model run. However, with a sensible naming style, this should not
  happen. To specify the parameter values and naming schema used, simply modify
  the values specified in the file (lines 15-18, currently) and the associated
  file path/name variables (lines 25, 27, 29, and 31, currently). Values are
  stored in space separated lists and encoded as strings currently. Each string
  in this list is then used to overwrite the current parameter in the
  \texttt{parameters.cpp} file, before the file is made, then run, with the
  associated file passed as input. Note that, as these values are simply passed
  directly into the \texttt{parameters.cpp} file, they must be valid
  \texttt{C++} syntax. Furthermore, the initial values must be known to the
  shell script. In other words, the values specified on lines 34 through 37 must
  match the current values in the \texttt{parameters.cpp} file, character for
  character. Unfortunately, currently this script is rather hacked together, and
  not especially elegant so it may be hard to understand. Feel free to reach out
  to me at
  \href{mailto:mattmcdermott8@gmail.com}{\nolinkurl{mattmcdermott8@gmail.com}}
  in case of confusion. 
\section{Current Algorithm}
\subsection{High Level Description}
The current algorithm is very straightforward. In particular, for each step in
time, stepping by a set parameter \texttt{tau} through \texttt{Duration}
minutes, the algorithm computes the net force and torque on the pronucleus and
then uses an eulerian approximation of the solution to the DE inspired by
friction dominated domain conditions to update the position and angle of
rotation of the body. The algorithm also maintains a list of endpoints of a set
number of MTs, updating these through growth, shortening, and respawning when
necessary. It also uses the same risk formulas for catastrophe and rescue as did
the previous matlab model. The only real differences are that several bugs are
corrected, force and torque are calculated upon every iteration directly, and
the implementation outputs final data to a file for another program to read in.
It is worth noting that the current implementation has a separate, abstracted
Vector class to handle vector addition, scalar multiplication, taking
projections and finding the norm. This makes the code more readable, but also
enables a much easier switch from 2D to 3D. In fact, the Vector class itself is
dimension independent (as in, it is set \texttt{Vector.hpp} but can be changed
without consequence) so moving to a 3D model really only amounts to modifying
the cortex positioning code. 
\subsection{Implementation Details}
\subsubsection{MT Envelope Restriction:}
In this code, there is a parameter which specifies the allowed envelope of
growth for MTs from either the mother MTOC or the daughter MTOC (separately).
This is accessed via the variable \texttt{envWidthM(D)} and
\texttt{envelopeM(D)}. For standard envelopes, simply setting the width should
be sufficient (as this actually is used in the \texttt{envelope} parameter) but
for stranger cases, it is also possible to just modify the \texttt{envelope}
parameter directly and set the envelope that way. We have tried and discussed a
number of ways to use this parameter, but have finally settled upon the idea
that the envelope merely limits growth. In particular, a new MT can only be
spawned within the allowed envelope (meaning spawned at an angle contained in
the range specified by the \texttt{envelope} parameter, with 0 degrees being the
angle of rotation of the MTOC anchor point relative to the pronucleus) but an MT
that falls outside of this envelope as the pronucleus rotates is allowed to
continue to grow, connect, and catastrophe as normal. If this needs to be
changed, see line 604 of \texttt{main.cpp}, near the comment beginning with
\texttt{MT-ENV}, which is just before the length-based respawning code. 
\subsubsection{Spring Implementations:}
Currently there is built in spring capabilities that can be activated with or
without pushing forces on either MTOC separately that do account for potential
contact and resulting force transfer with the pronucleus. The springs are turned
on or off in the two boolean parameters \texttt{motherSpringOn} and
\texttt{daughterSpringOn} in the file \texttt{parameters.cpp}. If a spring is
on, then the motion of that MTOC is governed by three things: the forces from
the associated MTs, the spring force between the MTOC and its anchor point
(using current displacement), and a random displacement term. Note that this is
altered slightly if the MTOC is close enough to the pronucleus. In that case,
the spring force directed towards the pronucleus is reflected by the pronucleus,
inducing a force on the MTOC away from the pronucleus and on the pronucleus away
from the MTOC. Note, however, that this \textbf{does not} imply that the MTOC
experiences a \emph{net} force away from the pronucleus. This is not a
reflection collision event; rather, the force from the MTOC towards the
pronucleus is effectively just \emph{transferred} to the pronucleus. The force
induced by the pronucleus on the MTOC is completely balanced out by the initial
force towards the pronucleus on the MTOC in the first place, resulting in only
net tangential forces on the MTOC after contact. Similarly, in this case, the
random displacement is only allowed to occur tangentially. Note that when the
spring is enabled, there is no \emph{direct} action on the pronucleus by the
MTs. Rather, the MTs induce a force on the MTOC, which moves, thus inducing a
spring force on the pronucleus (and itself, in opposite magnitude) the next
timestep. Thus, the pronucleus is only ever directly moved by the spring with
this model. 
\subsubsection{Boundary Checking:}
The current system checks for a cortex pronucleus collision and a MTOC
pronucleus collision using slightly different methods. For the MTOC pronucleus
collision, the system simply determines if the MTOC is within \texttt{Prad +
0.1} of the pronucleus position, using simple vector arithmetic. For the
pronucleus cortex collision, the system must do something slightly stranger.
Here, the system determines if the center of the pronucleus has impacted a
slightly smaller ellipse than the cortex. This doesn't give exactly the desired
behavior, as the curvature makes this slightly inaccurate, but in practice it is
quite effective. 
\subsubsection{MT Connection Density Limitation:}
The system maintains a collection of windows, within each is only allowed one
contact. The contact windows are computed in mathematica, and stored in the
\texttt{parameters.cpp} file as plain values. There is another, finer set of
windows which is commented out in this file as well. 
\subsubsection{Band Specification:}
The band specification is very straightforward. A list of fixed cortex angles
are stored, beginning with $0$ and ending with $2\pi$. These specify the
endpoints of all ``cortex pieces,'' which are angular, uniform sectors of the
cortex. This is stored in \texttt{regionAngles}. The probability of contact in
these areas and force multipliers scaling contacts in these areas are stored in
\texttt{regionProbabilities} and \texttt{regionForceMultipliers}, respectively.
These are each one component shorter than \texttt{regionAngles}, as index
\texttt{i} of the probability and multiplier vectors describe the effect on the
region between angles \texttt{regionAngles[i]} and \texttt{regionAngles[i+1]}. 
\subsubsection{Other Miscellaneous Notes:}
\begin{itemize}
  \item It is important to the current implementation that \texttt{mt\_numb} is
    a constant value, as otherwise fixed-size arrays could not be used to store
    all the codes data. 
  \item The \texttt{parameters.cpp} file defines not only fixed constants for
    the program, but also initializes global variables, a few  import
    \texttt{typedef}s, and an \texttt{enum} that defines the centrosome options.
    The first typedef sets up a new \texttt{float\_T} type, which is used so
    that precision of the model can be altered painlessly, and the second wraps
    the separate \texttt{Vector} class type with a \texttt{vec\_T} type, which
    is actually not needed anymore and could be removed, so long as all the code
    was updated to use \texttt{Vector} instead of \texttt{vec\_T}. 
  \item The two \texttt{operator} functions at the top of \texttt{main.cpp} are
    merely to aid in file output. 
  \item The net force calc function is simple; It merely sums up the all the force
    directions for the MTs, then multiplies that sum by the force for any given
    MT. This will give the net force on that centrosome. 
  \item The net force function is just a wrapper around net force calc. 
  \item The \texttt{updatePNPos()} function is an extraction of the code to
    update the position of the pronucleus from the main loop. The
    actual mechanics of the function are very straightforward currently, and
    simply translate the components of the force into torque and motion
    inspiring forces, respectively. 
  \item \texttt{advanceMT} is simply an encapsulation of some messy code from
    the loop. It grows or shrinks MTs. 
  \item \texttt{respawnMT} creates a new MT. 
  \item The main loop appropriately orders these functions, updates contact
    statistics, growth statistics, and growth velocities. It also checks for
    rescue and catastrophe and writes the data to the file. 
  \item Note here that a major implementation detail between this algorithm and
    the matlab one is that as this algorithm computes the force on every
    iteration, not just on a new contact, the \texttt{MT\_touch} array must be
    active every second, which in particular means that it cannot be used to
    distinguish between shortening velocities as it is in the matlab model. So,
    instead, an additional array is maintained, \texttt{MT\_GrowthVel\_*}, where
    * is either M or D. This stores the current growth velocity of the given MT. 
\end{itemize}
\section{Future Work}
\begin{enumerate}
  \item Add generic, extendable cortex positioning code. This would ensure a
    smooth change from 2D to 3D. Test and optimize the code. No performance
    testing or benchmarking has been done to date. 
  \item Test Stability
\end{enumerate}
\end{document}
